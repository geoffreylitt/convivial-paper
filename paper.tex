% -*- coding: utf-8; -*-
% vim: set fileencoding=utf-8 :
\documentclass[english,submission]{programming}
%% First parameter: the language is 'english'.
%% Second parameter: use 'submission' for initial submission, remove it for camera-ready (see 5.1)

\usepackage[backend=biber]{biblatex}
\addbibresource{references.bib}

\begin{document}

\title{Wildcard: Browser Extensions for All}
\subtitle{}% optional

\author{Geoffrey Litt}
\authorinfo{is bla bla bla}
\author{Daniel Jackson}
\authorinfo{is bla bla bla}
\affiliation{Massachusetts Institute of Technology}

\keywords{programming journal, paper formatting, submission preparation} % please provide 1--5 keywords


%%%%%%%%%%%%%%%%%%
%% These data MUST be filled for your submission. (see 5.3)
\paperdetails{
  %% perspective options are: art, sciencetheoretical, scienceempirical, engineering.
  %% Choose exactly the one that best describes this work. (see 2.1)
  perspective=art,
  %% State one or more areas, separated by a comma. (see 2.2)
  %% Please see list of areas in http://programming-journal.org/cfp/
  %% The list is open-ended, so use other areas if yours is/are not listed.
  area={Social Coding, General-purpose programming},
  %% You may choose the license for your paper (see 3.)
  %% License options include: cc-by (default), cc-by-nc
  % license=cc-by,
}
%%%%%%%%%%%%%%%%%%

%%%%%%%%%%%%%%%%%%
%% These data are provided by the editors. May be left out on submission.
%\paperdetails{
%  submitted=2016-08-10,
%  published=2016-10-11,
%  year=2016,
%  volume=1,
%  issue=1,
%  articlenumber=1,
%}
%%%%%%%%%%%%%%%%%%

%%%%%%%%%%%%%%%%%%%%%%%%%%%%%
% Please go to https://dl.acm.org/ccs/ccs.cfm and generate your Classification
% System [view CCS TeX Code] stanz and copy _all of it_ to this place.
%% From HERE
\begin{CCSXML}
<ccs2012>
<concept>
<concept_id>10002944.10011122.10003459</concept_id>
<concept_desc>General and reference~Computing standards, RFCs and guidelines</concept_desc>
<concept_significance>300</concept_significance>
</concept>
<concept>
<concept_id>10010405.10010476.10010477</concept_id>
<concept_desc>Applied computing~Publishing</concept_desc>
<concept_significance>300</concept_significance>
</concept>
</ccs2012>
\end{CCSXML}

\ccsdesc[300]{General and reference~Computing standards, RFCs and guidelines}
\ccsdesc[500]{Applied computing~Publishing}

% To HERE
%%%%%%%%%%%%%%%%%%%%%%%

\maketitle

% Please always include the abstract.
% The abstract MUST be written according to the directives stated in 
% http://programming-journal.org/submission/
% Failure to adhere to the abstract directives may result in the paper
% being returned to the authors.
\begin{abstract}

\end{abstract}


\hypertarget{introduction}{%
\section{Introduction}\label{introduction}}

This is the introduction of the paper.

\hypertarget{demos}{%
\section{Demos}\label{demos}}

\acks
\printbibliography

\end{document}

% Local Variables:
% TeX-engine: luatex
% End:
